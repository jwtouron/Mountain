% Created 2021-08-03 Tue 11:06
% Intended LaTeX compiler: pdflatex
\documentclass[11pt]{article}
\usepackage[utf8]{inputenc}
\usepackage[T1]{fontenc}
\usepackage{graphicx}
\usepackage{grffile}
\usepackage{longtable}
\usepackage{wrapfig}
\usepackage{rotating}
\usepackage[normalem]{ulem}
\usepackage{amsmath}
\usepackage{textcomp}
\usepackage{amssymb}
\usepackage{capt-of}
\usepackage{hyperref}
\author{Pradyun Narkadamilli}
\date{\today}
\title{Mountain Syntax Highlighting Specifications}
\hypersetup{
 pdfauthor={Pradyun Narkadamilli},
 pdftitle={Mountain Syntax Highlighting Specifications},
 pdfkeywords={},
 pdfsubject={},
 pdfcreator={Emacs 27.2 (Org mode 9.5)}, 
 pdflang={English}}
\begin{document}

\maketitle
\tableofcontents



\section{Introduction}
\label{sec:org7515805}
\begin{quote}
Disclaimer: As the Hotaka variant of Mountain becomes more developed, these syntax specifications will be modified to have exceptions/differences specifically meant for Hotaka. In the meantime, assume that Hotaka and Fuji use the same standards.
\end{quote}

\subsection{Formatting this specifications}
\label{sec:org2dfd4a2}
This document is modeled after the perennial Dracula theme. In similar fashion, Language syntaxes and scopes will be formatted as such:
\begin{quote}
\emph{Object - (ForegroundColor, BackgroundColor\textsubscript{opt}) Italicize\textsubscript{opt} Bold\textsubscript{opt}}
\end{quote}
Throughout this document, syntax tokens will generally be using the TextMate Naming Conventions.

\subsection{Color Palette}
\label{sec:orgffc3423}
The colors listed below are in line with the \emph{Fuji} color palette.
\begin{center}
\begin{tabular}{lll}
Name & Color & Hex\\
Yoru & Background & \#0F0F0F\\
Yuki & Foreground & \#F0F0F0\\
Ume & Purple & \#8F8AAC\\
Kosumosu & Pink & \#AC8AAC\\
Chikyu & Yellow & \#ACA98A\\
Aki & Orange & \#C6A679\\
Mizu & Cyan & \#8AACAB\\
Take & Green & \#8AAC8B\\
Shinkai & Blue & \#8A98AC\\
\end{tabular}
\end{center}
\end{document}
